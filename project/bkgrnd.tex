%
% bkgrnd.tex
%

There is a wealth of studies surrounding the usability of applications. Three approaches are described below.
\section{Evaluation Matrix}
Suzanna Conrad and Julie Shen suggest that a usability study should be conducted as follows: an exploratory survey is held to begin the studies, followed by evaluation matrix testing (which gives a website scores based on usability), and three rounds of formal usability testing \cite{usercen}. A formal usability study is an observation of a user using the target product, usually by one or more facilitators taking notes on the user's process and explaining which tasks the user must attempt to complete \cite{usercen}. An evaluation matrix is developed based on the initial survey, reflecting the interests of the target user group \cite{usercen}. Each interest is given a weight based on a point system \cite{usercen}. 
\section{Goal Focused Usability Studies}
Brenda Battleson, Austin Booth, and Jane Weintrop \cite{casestudy} present the idea that goals for the product must be set before any testing or planning begins. These goals then become the basis for the questions used in formal usability studies \cite{casestudy}. Studies suggest a planned list of tasks for the user to complete along with post test discussion questions for the user to discuss how they felt using the product \cite{casestudy}. 
\section{Goal-Oriented Design}
Alan Cooper, Robert Reimann, David Cronin, and Christopher Noessel \cite{aboutface} suggest a goal-oriented view on the development of products\cite{aboutface}. Generally, products are developed based on the wants and needs of the developers of the product, but the goal-oriented design process attempts to eliminate this problem \cite{aboutface}.  Usability testing is used to test the product after it is complete, but not generally to guide the design of a product \cite{aboutface}. In goal-oriented design, there are six phases for the design of a product: Research, Modeling, Requirements Definition, Design Framework, and Design Refinement \cite{aboutface}. The "Research" phase consists of researching existing products, stakeholder interviews where the designer will obtain the stakeholder's vision for the product along with business related information, and user interviews and observation. This phase is the foundation for every other phase in the design process \cite{aboutface}. User interviews should take place where the product interaction normally happens \cite{aboutface}. Facilitators should avoid a fixed set of questions, assume the role of an apprentice (not an expert), use open-ended and closed-ended questions to direct the discussion, focus on goals first and tasks second, avoid making the user a designer, avoid discussing technology, encourage storytelling, ask for a show-and-tell, and avoid leading questions \cite{aboutface}. Table 2.1 demonstrates the types of questions that will be most helpful during user interviews.\\

\begin{table}
	\caption{User Interview Questions \cite{aboutface}}
	\begin{tabular}{ | m{15em} | m{8cm}| } 
		\hline
		Question Type & Question\\ [.05ex] 
		\hline\hline
		Goal Oriented Questions & What makes a good day? A bad day? 
		What activities currently waste your time? 
		What is most important to you? 
		What helps you make decisions?\\ 
		\hline
		System Oriented Questions & What are the most common things you do with the product? 
		What parts of the product do you use most?
		How do you work around problems?
		What shortcuts do yo employ?\\ 
		\hline
		Work Flow Oriented Questions & What did you do when you first came in today? What did you do after that? 
		How often do you do this? What things do you do weekly or monthly, but not every day?
		What constitutes a typical day? What would be an unusual event? \\
		\hline
		Attitude Oriented Questions & What do you see yourself doing five years from now?
		What would you prefer not to do? What do you procrastinate on?
		What do you enjoy most about your job (or lifestyle)? What do you always tackle first?\\ 
		\hline
	\end{tabular}
\end{table}

Phase two, "Modeling", is where the designer will produce 'Personas' \cite{aboutface}. These personas are user archetypes based on patterns in user and customer behaviors, attitudes, aptitudes, goals, environment, tools, and challenges \cite{aboutface}. A persona represents a range of the potential users \cite{aboutface}. These are sorted into persona types and one primary persona is chosen for each of the primary interfaces of the product \cite{aboutface}. The next phase is the "Requirements Definition" phase. In this phase, the designer constructs scenarios where their primary persona would use the product and then compiles this scenario into the product requirements, which include the functional and data needs, user mental models, design imperatives, product vision, business requirements, and technology requirements \cite{aboutface}. The fourth phase is the "Design Framework" phase where the designer constructs rough frameworks for the interfaces of the product then uses 'key path' scenarios which test how the developed persona would interact with the product \cite{aboutface}. The fifth phase is the "Design Refinement" phase, where the design becomes more detailed and takes a more concrete form, including visualization and branding \cite{aboutface}. The last phase is the "Design Modification" phase. This phase is concerned with maintaining the conceptual integrity of the design under changing technology constraints \cite{aboutface}. These phases make up the process called 'Goal-Directed Design' and aims to generate products that are usable, viable, and build-able \cite{aboutface}. Digital products often fail because the developers have misplaced priorities, are ignorant about real users, lack a design process, or suffer from a conflict of interest by performing the designing and the developing \cite{aboutface}. If designers stick to the concepts behind Goal-Directed Design and use this thinking through-out the development of products, designers can create products that will "surpass the competition, make devoted fans of their users, and - perhaps - make the world a better place, one pixel at a time" \cite{aboutface}.