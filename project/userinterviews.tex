%
% userinterviews.tex
%
\subsection{User Interview No. 1}
Age: 21\\
School Status: Undergraduate Student at Appalachian State University \\
Major: Computer Science\\ \\
Q: Do you have a planner?\\
A: Yes \\ \\
Q: How often do you look at your planner?\\
A: Now, 3-4 times a week, but last year when I was really busy it was like my life.\\ \\
Q: Do you use the calendar or the day to day list?\\
A: Both, I really like seeing a birds eye view of everything on my calendar, but I also like to plan out my days with lists. When I get home I have a whiteboard on my wall where I transfer everything I need to do that day so I can keep seeing it all day.\\ \\
Q: When you plan your days, do you plan it at the start of the week or the each day?\\
A: I could never plan out my days by week, it’s always by day.\\ \\
Q: Do you ever schedule your days out, like by time?\\
A: Rarely \\ \\
Q: Do you think you would like to see reports, like if one class is taking up more of your time?\\
A: Yeah, I think that would be really cool.\\ \\
Q: Have you ever tried online planners? \\
A: No, not really.\\ \\
Q: Why?\\
A: I'm an old school person I guess. If I don't write things down I won't remember them.\\ \\
Q: If you found an digital planner that offered capabilities that you really liked, do you think you would switch?\\
A: Yeah, especially if it was connected to my google drive. I find that I’ve stopped using applications not on my google drive. If it was accessible on my phone that would pretty awesome too.\\ \\
Q: Have you used Trello? \\
A: Yes, I was forced to use it for a class.\\ \\
Q: So I take it that you didn't like it? \\
A: No, well, I really like the process aspect of it. I like that I can put things into ‘in progress’ and then moving things to the done column makes me feel really accomplished. Like marking things off my list or doing some kind of action when I finished things makes me feel good. Checking a box is less satisfying. The thing about Trello is that I really like seeing a birds eye view of everything. It’s good for day to day planning, but not for a whole planner. \\ \\
Q: So you would like it better if it had a calendar? \\
A: Yeah, for sure.\\ \\
*The subject was then directed to MyStudyLife with test credentials and then instructed to click around and get a feel for the application, thinking aloud.*\\
She saw the dashboard, and was confused what to click on at first. She tried to add a task to the task list. and was slightly overwhelmed by the options. She liked that you could pick a color for the subject of the task. \\
After creating a task, she saw that to mark a task as complete she would have to move a percentage completed bar to 100\%. She did not like trying to apply a percentage to a task, also stated that it was not satisfying enough when done.\\
She stated multiple times that the application was not intuitive.\\
She liked the month view, but the month view doesn’t show tasks on days. She could only see a task for a specified day by clicking on that day. She did not like this.\\
She saw that there was a task list view, but did not like the way it was setup or that you still had to use the percentage bar to mark a task as complete.\\
She clicked through other parts of the dashboard and didn’t find much that she liked about the other functionalities, like she found the search feature unnecessary.
\\ *The subject was then directed to WeekPlan with test credentials.*\\
She was immediately confused and overwhelmed.\\
She did not like the terminology or the 24hr format.\\
She also did not like that it showed days by time.\\
She found really nothing she liked about this site except for the analytics. She thought it would be cool to see analytics, but possibly depressing.\\

\subsection{User Interview No. 2}
Age: 22\\
School Status: Undergraduate Student at Appalachian State University \\
Major: Mathematics\\ \\
Q: Do you have a planner?\\
A: Yes \\ \\
Q: How often do you look at your planner?\\
A: Every day\\ \\
Q: I see that you use google calendar, why did you choose an online planner?\\
A: I can always access it from anywhere.\\ \\
Q: Is there capabilities you wish you had with Google Calendar?\\
A: I wish I could see a task list with due dates and also if there’s a long list for a day sometimes the calendar view gets kind of messy to look at.\\ \\
Q: Do you ever schedule your days out, like by time?\\
A: Sometimes, usually on pencil and paper. \\ \\
Q: If you found another online application that gave you these capabilities would you switch?\\
A: Yeah for sure. I tried some other online planners, but I didn’t find any that I liked.\\

*The subject was then directed to Trello with test credentials and then instructed to click around and get a feel for the application, thinking aloud..*\\
\\
He immediately said "Oh this is cool."\\
He liked all the options for tasks, although voiced the need for recurring tasks.\\
He liked the ability to label a task with a class/color, which is not currently available with his planner.
Overall, he liked Trello, but really wanted to be able to see a calendar view.

*The subject was then directed to MyStudyLife with test credentials.*\\
He immediately reacted well to the abundance of features. He liked the calendar and task views and though that he would use this planner.\\
He attempted to make a task and exam.\\
He voiced annoyance with the inability to see task on the calendar and inability to edit from the calendar.\\
Overall, he thought this planner was close to what he needed, but had extra features he didn't need and needed to expand on the calendar view.

*The subject was then directed to WeekPlan with test credentials.*\\
He commented that the interface was not intuitive.\\
He liked the animation that happens when a task is checked.\\
Overall, the interface was too annoying and messy for him.

\subsection{User Interview No. 3}
Age: 22\\
School Status: Graduate Student at University of North Carolina at Chapel Hill\\
Major: Social Work\\ \\
Q: Do you have a planner?\\
A: Yes \\ \\
Q: How often would you say you look at your planner?\\
A: Well it depends. When I’m busy I look at it every day, multiple times a day, but when I’m not busy I would say I look at it 2 to 3 times a week.\\ \\
Q: Do you use just a calendar or a day by day list?\\
A: Both, I keep track of what I need to do each day on the day by day list. If I’m really busy I might make a schedule for that day, but mostly it’s just a list of things I need to do each day and my calendar.\\ \\
Q: When you plan out what needs to be done each day? Do you plan each day out as you go? \\
A: I plan out my weekly stuff, but mostly I plan out for each day what needs to be done that day. \\ \\
Q: Do you think it would be helpful for you to see reports like how much time or percentage of tasks you spend on each class? \\
A: Yeah actually, that could be kind of nice to know. \\ \\
Q: Have you ever tried online planners? \\
A: Nope, the closest would be google calendar, but I don’t use that as my planner. \\ \\
Q: Why haven’t you tried any online planners? \\
A: I am an in person kind of person. I like to hold my planner and be able to write on it and see it. I haven’t ever thought about trying online planners. \\ \\
The subject opted to try out each website for a week and then give her feedback.
\\After the week was up she stated that she really liked Trello, but wished it had a calendar functionality. She really valued the ability to plan week by week. MyStudyLife had too many functionalities and was overcomplicated for her needs and the same for WeekPlan.

\subsection{User Interview No. 4}
Age: 24\\
School Status: Nursing student at Asheville-Buncombe Technical Community College\\
\\
Q: Do you have a planner?\\
A: Yes, I love planners. \\ \\
Q: How often do you look at your planner?\\
A: Every day\\ \\
Q: Do you use the calendar or the day by day list?\\
A: The day by day lists for sure, I don’t have much written on the calendar part. I have everything color coded too.\\ \\
Q: Do you ever schedule your days out, like by time?\\
A: Sometimes, but not in my planner.\\ \\
Q: When you plan out your days, do you do it each day or at the start of the week?\\
A: At the start of the week I plan out my day by day to do lists. I could never do it day to day, that makes me anxious.\\ \\
Q: Do you think reports, like how much of your tasks each class is consuming would be helpful?\\
A: No, I don’t think so.\\ \\
Q: Have you tried online planners?\\
A: Yes, I haven’t seen any I like. They all kind of suck, but I would love to switch for accessibility purposes.\\ \\

*The subject was then directed to Trello with test credentials and then instructed to click around and getting a feel for the application, thinking aloud.*\\

She immediately thought that the interface looked "cool". She liked that you can have a checklist and stated that she is "all about checklists." She was not sure what the attachments feature was for, but still thought it was cool.\\
Overall, she thought Trello was pretty cool. She liked the list features and the process of moving tasks over and being able to see upcoming tasks.

*The subject was then directed to MyStudyLife with test credentials.*\\
"Ooh this looks fun," she exclaimed. She liked that this one provided a calendar view with time slots, however she thought it was weird that you can't click on things in the calendar or view tasks.\\
She liked the task view.\\
Overall, she liked the functionalities of this planner, but felt it could be laid out differently and refined. She felt that there wasn't a great way to put tasks to days that aren't due dates and wishes there was a way to plan out what she would like to complete each day.

*The subject was then directed to WeekPlan with test credentials.*\\
Immediately, she felt the interface was confusing and strange. She did not wish to continue playing with this planner.

\subsection{User Interview No. 5}
Age: 21\\
School Status: Sophomore at Appalachian State University\\
Major: Computer Science
\\
Q: Do you have a planner?\\
A: Yes. \\ \\
Q: How often do you look at your planner?\\
A: 0-5 times a day depending on the workload.\\ \\
Q: Do you use the calendar part of your planner or the day by day?\\
A: Day by day and I use the calendar on my phone.\\ \\
Q: Do you plan out your days, like by time?\\
A: When I have a heavy schedule I do.\\ \\
Q: Have you ever tried an online planner?\\
A: I tried a couple and used on in the terminal for a while, but I didn't really like any of them.\\ \\

*The subject was then directed to Trello with test credentials and then instructed to click around and getting a feel for the application, thinking aloud.*\\
He clicked around silently for some time and then came to a quick conclusion that if he were going to use a digital planner, it would be like this one. He stated that he liked this layout, but he didn't feel like it would make him switch to a digital option.

*The subject was then directed to MyStudyLife with test credentials.*\\
As with Trello, he clicked around silently for a few minutes and then came to another quick conclusion that he liked Trello better as it was simpler.

*The subject was then directed to WeekPlan with test credentials.*\\
He once again clicked around for some time and then gave his thoughts. He did not like this interface, but felt that if it were mixed with Trello he would possibly use it.

\subsection{User Interview No. 6}
Age: 23\\
School Status: Graduate Student at Appalachian State University\\
Major: Computer Science
\\
Q: Do you use a planner?\\
A: Yes, google calendar. \\ \\
Q: How often do you look at your calendar?\\
A: Every hour.\\ \\
Q: So I take it you prefer the calendar view?\\
A: I like to look at each week, but not necessarily the whole month.\\ \\
Q: Do you plan out what you are going to do each day?\\
A: Not really.\\ \\
Q: Do you think reports would be helpful to see?\\
A: Yeah, might be hard to implement correctly though since most of the time I'm working on something I take breaks as well.\\ \\
Q: Have you tried any other online planners?\\
A: Like freshman year, but I can't remember which ones.\\ \\

*The subject was then directed to Trello with test credentials.*\\
So click around and get a feel for the application. Try to say your thoughts out loud.\\
He initially thought the site looked "pretty cool". He asked if you can sort by due date and I directed him how to do that.\\
He really liked the options on each card and due date abilities, but really would like a calendar with a weekly view and times.

*The subject was then directed to MyStudyLife with test credentials.*\\
He immediately found the calendar and liked that it had that capability.\\
He liked that you can put your full schedule of classes, but noticed that you can't view tasks on the calendar without clicked on individual days.

*The subject was then directed to WeekPlan with test credentials.*\\
He clicked around for a bit, silently. He then said, "I don't know. This is pretty weird."

\subsection{User Interview No. 7}
Age: 22\\
School Status: Senior at Appalachian State University\\
Major: Computer Information Systems
\\
Q: Do you have a planner?\\
A: I try but fail, so nah, I stick to notifications and reminders on my phone.\\ \\
Q: Do you use a specific app or just reminders through the calendar?\\
A: Both the calendar and the reminders app on iOS\\ \\
Q: Have you ever tried any online planners or planner apps?\\
A: I haven’t!\\ \\
Q: Do you think you would use an online planner if you found one that met your needs?\\
A: I think I would! Probably when I get a more portable laptop.\\ \\
Q: Do you like to keep track of just due dates or do you like to plan out what tasks you’re going to do on which days?\\
A: A little of both, sometimes due dates, sometimes medication reminders, and appointment/meeting reminders. Or sometimes it’s even just to remind me to respond to emails if it’s important.\\ \\

*The subject was then directed to Trello with test credentials and then instructed to click around and getting a feel for the application, thinking aloud.*\\

She liked the interface. She really liked the labeling features where you can "create your own recognizable scheme".\\
She liked that tasks can be as detailed as you want them to be and that it's easy to sort between activities that are done, in progress, or need to be started so it gives "a sense of accomplishment".\\
Overall, she felt like she would use the site, but wishes there was an ability to have notifications possibly on a task by task basis.

*The subject was then directed to MyStudyLife with test credentials.*\\
She immediately loved the interface of this planner. She liked the ability to adjust percentage bars on tasks and the ability to keep all academic related items in one place, even exams.\\
She felt that this planner was more academically based and wouldn't be as good for non-academic tasks.\\
She stated that she wished she had known about this earlier this year.

*The subject was then directed to WeekPlan with test credentials.*\\
She liked the different sections like the week planner, high impact tasks, and parking lot section. She liked the ability to keep non-academic related tasks separately and felt this planner was good for planning and goal setting.\\
Overall, she felt this site was more busy and a bit overwhelming. She thought the site could get cluttered if you have a lot on it.

A little of both, sometimes due dates, sometimes medication reminders, and appointment/meeting reminders. Or sometimes it’s even just to remind me to respond to emails if it’s important.

\subsection{User Interview No. 8}
Age: 21\\
School Status: Junior at Appalachian State University\\
Major: Nutrition
\\
Q: Do you have a planner?\\
A: Yes\\ \\
Q: Is it a paper planner?\\
A: Yes, but I've been really lax about using it lately, I really need to get better about that.\\ \\
Q: Have you ever tried any online planners?\\
A: I tried google calendar and windows calendar, but I don't need to keep a schedule so I stopped using those.\\ \\
Q: So do you use the day by day portion of your planner over the calendar?\\
A: Yes, my planner has 2 weeks on a page and I write stuff for each day.\\ \\
Q: So do you plan out what you are going to do each day?\\
A: Yes, at the beginning of the week.\\ \\

*The subject was then directed to Trello with test credentials and then instructed to click around and getting a feel for the application, thinking aloud.*\\

She liked the columns, but also not sure about using them in practice. She thought that it would take time to set up to her liking, but likes that it's personalized. She asked if there is a mobile app and I answered, "yes".\\
Overall, she would probably try out this planner, but she would like it if it was an app on her computer better. She also really liked the color coding options and the ability to connect to google.

*The subject was then directed to MyStudyLife with test credentials.*\\
Her immediate response was that this site seemed more user friendly than Trello. She couldn't figure out what "resit" meant on the exam creation form. She also thought that the calendar was weird, but liked that buttons told you what action they performed when she hovered the mouse over them.\\
She liked the to do list better on Trello, but liked the ability to mark percentages on tasks and organize by subject on this site. She also liked the exam piece, but felt it could be treated as a task.\\ 
Overall, she decided she liked Trello better. She felt MyStudyLife put extra pressure on her with the circles/percentages on the dashboard.

*The subject was then directed to WeekPlan with test credentials.*\\
She clicked around for a while silently. She exclaimed when she check off a task and there was an animation and mentioned that the animation was a great incentive for completing a task, however she felt the application as a whole had too much going on.\\
She felt this site would be better for work or maybe collaboration. She also mentioned that the days are tiny and have tiny text that is hard to read. \\
Overall, she did not like the layout of this site.